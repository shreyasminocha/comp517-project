\documentclass[twocolumn]{article}

\usepackage{hyperref}
\usepackage{lipsum}
\usepackage{../usenix-2020-09}

\title{COMP 417 Project Proposal}
\author{John Cesarz \& Shreyas Minocha}

\begin{document}

\maketitle

\section{Introduction}

% 1. Problem + why it's important
There is often a need to publicly archive valuable data in a manner that is resistant to accidental data loss, owing to either system failure or maintainer neglect.
% TODO: motivate further. tug at heartstrings.
%
% 2. Research context or gap in existing approaches
One common approach to sharing data within a community with redundancy is to use peer-to-peer networks like BitTorrent.
However, BitTorrent-based file sharing suffers from risks of data loss when the number of peers sharing a torrent is low.
BitTorrent has no mechanism for triggering the replication of ``at-risk files'', instead relying on the users of the network to coordinate this.
However, in practice, at-risk files are often neglected, until the last few peers sharing the file disappear, at which point the file is lost.
% TODO: talk about IPFS
%
% 3. The aim/goal/hypothesis (typically a single sentence, but can be longer)
Our goal is to find ways to find ways to prevent the problem of data loss in peer-to-peer file sharing networks by cleverly triggering replication of at-risk files.
%
% 4. The proposed solution: how will you solve it?
% TODO
%
% 5. How you will evaluate it?
%
We will evaluate our solution on the basis of its effectiveness at avoiding data loss, the storage overhead introduced from the replication, and the network overhead incurred by transferring file chunks for redundancy.
%
% 6. Anticipated Contributions
We expect to build a prototype of a distributed file system that performs well under these metrics, offering a competitive alternative for long-term sharing and archival of important data.

\section{Background}

% Detail a few specific elements of the problem that are necessary to understand the work.
% This can be an expanded version of the problem as mentioned in the introductory paragraph.

Our distributed filesystem will be optimized for archival use.
In such a system, new data is frequently added, but existing data is rarely modified or deleted.
For archival systems, data generally needs to be stored for long periods of time, much longer than the lifetime of the typical disk.
As a result, redundant copies of data need to be maintained in order to support data integrity in the event that a single disk fails.

Making a filesystem distributed also introduces several unique problems.
Potential consumers of a particular piece of data need to be able to determine which nodes in the network contain that data and request it.
The lack of a central node also complicates the process for maintaining redundancy in the event of a node's failure.

\section{Methods and Plan}

% Detail the early design of the solution.
% Also describe how you plan to evaluate it.

We will use a Merkle tree to represent the file system hierarchy.
This will give us a way to easily achieve content-addressable storage, and to detect changes to the file system.

% What will you need to accomplish your goal?
To efficiently detect resources that have a low number of peers, we will have each peer in the network track the number of peers that also have a given resource.
When this number drops below a threshold, the peer will attempt to trigger network-wide replication of the resource.
Our system will determine that threshold based on heuristics that factor in the size of the resource, the number of peers in the network, and the number of peers that have the resource.
We may also combine error-checking codes with the replication as a mechanism for network-wide redundancy.
Error-checking codes will allow us to avoid data loss more efficiently than replication would.

We will evaluate our system by running it on a network of peers, and simulate peer loss.
We will use BitTorrent and IPFS as baselines for comparison.

\section{Milestones}

% Itemize a list of deadlines that allow for incremental role out of the work.

\begin{itemize}
	\item 2023-11-01: Represent a file-system hierarchy in a content-addressable manner.
	\item 2023-11-05: Connect to peers and exchange file chunks.
	\item 2023-11-08: Implement error-correcting codes for redundancy.
	\item 2023-11-12: Trigger replication of at-risk files.
	\item 2023-11-15: Refine replication thresholds and heuristics.
	\item 2023-11-19: Evaluate ability to avoid data loss against baselines.
	\item 2023-11-22: Evaluate storage and network overhead against baselines.
\end{itemize}

\end{document}
