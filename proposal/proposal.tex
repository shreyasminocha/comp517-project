\documentclass{article}

\usepackage{lipsum}

\title{COMP 417 Project Proposal}
\author{John Cesarz \& Shreyas Minocha}

\begin{document}

\maketitle

\section{Introduction}

% 1. Problem + why it's important
There is often a need to archive valuable public data in a manner that is resistant to both active censorship and accidental data loss, owing to either system failure or maintainer neglect.
% TODO: motivate further. tug at heartstrings.
%
% 2. Research context or gap in existing approaches
One common approach to sharing data while resisting censorship is to use peer-to-peer networks like BitTorrent.
% TODO: insert bittorrent flaws that we hope to deal with
Additionally, BitTorrent traffic, for example, is pretty distinct and lends itself to rule-based censorship techniques.
% TODO: talk about IPFS
%
% 3. The aim/goal/hypothesis (typically a single sentence, but can be longer)
Our goal is to find ways to find ways to make file access more economical and reliable by adapting to the conditions of the file-sharing network.
%
% 4. The proposed solution: how will you solve it?
% TODO
%
% 5. How you will evaluate it?
%
We will evaluate our solution on the basis of its speed of data transfer, resistance to active censorship attempts, and the longevity of data under simulated scarcity of peers.
%
% 6. Anticipated Contributions
We expect to build a prototype of a distributed file system that performs well under these metrics, offering a competitive alternative for long-term sharing and archival of important data.

\section{Background}

% Detail a few specific elements of the problem that are necessary to understand the work.
% This can be an expanded version of the problem as mentioned in the introductory paragraph.

\lipsum[2]

\section{Methods and Plan}

% Detail the early design of the solution.
% Also describe how you plan to evaluate it.

% What will you need to accomplish your goal?

\lipsum[3]

\section{Milestones}

% Itemize a list of deadlines that allow for incremental role out of the work.

\lipsum[4]

\end{document}
